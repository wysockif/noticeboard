\documentclass{article}

\usepackage[utf8]{inputenc}
\usepackage[T1]{fontenc}
\usepackage{polski}
\usepackage{indentfirst}
\usepackage{lastpage}
\usepackage{natbib}
\usepackage{graphicx} 
\usepackage{sidecap}
\usepackage{wrapfig}
\usepackage{subfig}
\usepackage{caption}

\usepackage{fancyhdr}
\pagestyle{fancy}
\fancyhf{}
\rhead{Franciszek Wysocki}
\rfoot{Strona \thepage \hspace{1pt} z \pageref{LastPage}}
\lhead{Spis treści}
\title{Specyfikacja funkcjonalna dla projektu pt. ,,Noticeboard''}
\author{}
\date{}

\begin{document}
\maketitle

\begin{flushright}
\par
\vfill
\par
{\fontsize{11}{11}\selectfont
    Wykonał: Franciszek Wysocki

    Sprawdzający: mgr inż. Paweł Zawadzki

    Data: 17.03.2021
}
\end{flushright}
\thispagestyle{empty}

\newpage

\tableofcontents

\newpage

\section{Cel dokumentu}
{\fontsize{12}{12}\selectfont
    Celem ninijeszego dokumentu jest przedstawienie funkcjonalności aplikacji webowej ,,Noticeboard''. Zostaną w nim opisane m.in: cel projektu, interakcje pomiędzy systemem a użytkownikiem, jak również same funkcje systemu.
}

\lhead{Cel dokumentu}

\section{Cel projektu}
{\fontsize{12}{12}\selectfont
    Celem projektu jest stworzenie prostego serwisu ogłoszeniowego - aplikacji webowej w architekturze klient-serwer. Będzie ona rozwiązywać problem ograniczonego zasięgu ogłoszeń na tradycyjnych tablicach czy słupach reklamowych.
    
    Serwis ten będzie umożliwiać m.in. rejestrację oraz logowanie do serwisu, wyświetlanie/filtrowanie stron z ogłoszeniami (bez logowania), jak również dodawanie/kasowanie/edycja ogłoszeń (po uwierzytelnieniu). 
    
    Aplikacja będzie także odpowiednio zabezpieczona - walidacja danych, hashowanie hasła, reakcja na błędy ze strony użytkownika.
}

\section{Użytkownik serwisu}

{\fontsize{12}{12}\selectfont
    Aplikacja będzie kierowana do każdej osoby, dlatego interfejs użytkownika będzie prosty, aby każdy mniej techniczny użytkownik nie miał problemów z jej obsługą. 
    
    W celu korzystania z serwisu użytkownik powinien miec zainstalowaną przeglądarkę internetową z włączoną obsługą JavaScriptu. Preferowane jest korzystanie z przeglądarek wspierających ES6.
}

\section{Dostępne funkcje w serwisie i ich opis}
{\fontsize{12}{12}\selectfont
    
\begin{itemize}
    \item Rejestracja:
    Strona rejestracji będzie posiadać pola na adres email, nazwę użytkownika, hasło i powtórzenie hasła oraz przycisk zatwierdzający operacje. Dane zostaną zwalidowane (np. złożoność hasła, unikalność nazwy użytkownika). Mogą pojawić się komunikaty o błędach.
    
    \item Logowanie: 
    Strona logowania będzie posiadać pola na nazwę użytkownika i hasło oraz przycisk zatwierdzający operację. Po zatwierdzeniu nastąpi weryfikacja istnienia użytkownika i zgodności hasła. Mogą pojawić się komunikaty o błędach.
    
    \item Wyświetlenie, wyszukiwanie i filtrowanie ogłoszeń:
    Na stronie głównej zostanie wyświetlona lista ogłoszeń z możliowścią ich filtrowania oraz wyszukiwania (paginacja i lazy loading). Mogą pojawić się komunikaty o braku znalezienia wyników.
    
    \item Dodawanie ogłoszeń:
    Strona na dodawanie ogłoszeń (dostępna po zalogowaniu) będzie posiadać pola na tytuł, opis, cenę i lokalizację oraz przyciski na załączenie zdjęcie, zatwierdzenie operacji i skasowanie ogłoszenia. Nastąpi walidacja danych (np. ceny) i mogą pojawić się komunikaty o błędach.
    
    \item Edytowanie i kasowanie ogłoszeń: 
    Strona na edytowanie ogłoszeń będzie podobna do strony do ich dodawania, z tą różnicą, że pola będą automatycznie wypełniane i pojawi się na niej dodatkowy przycisk służący kasowaniu ogłoszenia.

    
\end{itemize}
}
\lhead{Przykładowy scenariusz użytkowania}

\section{Przykładowy scenariusz użytkowania}
{\fontsize{12}{12}\selectfont

Uruchomienie aplikacji nie będzie wymagało działania użytkownika, gdyż ta w końcowej wersji będzie stale uruchomiona na serwerze - więcej szczegółów w specyfikacji implementacyjnej.

\begin{itemize}
    \item Użytkownik zakłada konto oraz loguje się:
        \begin{enumerate}
            \item Użytkownik otwiera przeglądarkę i przechodzi pod odpowiedni adres URL.
            \item Użytkownik przechodzi na stronę rejestracji i wprowadza wymagane dane.
            \item Użytkownik zostaje poinformowany o pomyślności operacji, po czym zostaje przekierowany na stronę logowania.
            \item Użytkownik loguje się.
            \item Użytkownik zostaje przekierowany na stronę główną.
        \end{enumerate}

    \item Użytkownik jest ogłoszeniodawcą.
            \begin{enumerate}
            \item Użytkownik otwiera przeglądarkę i przechodzi pod odpowiedni adres URL.
            \item Użytkownik przechodzi ze strony głównej na stronę logowania i uwierzytelnia się, po czym zostaje z powrotem przekierowany na stronę główną.
            \item Użytkownik naciska przycisk ,,Dodaj ogłoszenie'' i podaje tytuł, opis, cenę, lokalizację, adres email oraz załadowuje zdjęcie, po czym zatwierdza operację.
            \item Użytkownik edytuje swoje (np. błędne) ogłoszenie.
            \item Użytkownik kasuje swoje (np. nieaktualne) ogłoszenie.
            \item Użytkownik wylogowuje się.
        \end{enumerate}
        
    \item Użytkownik jest ogłoszeniobiorcą.
        \begin{enumerate}
            \item Użytkownik otwiera przeglądarkę i przechodzi pod odpowiedni adres URL.
            \item Użytkownik przegląda i filtruje ogłoszenia.
            \item Użytkownik otwiera ogłoszenie i wyświetla szczegółowy opis, dane sprzedawcy i jego adres email do kontaktu.
        \end{enumerate}
    \end{itemize}
    
}

\lhead{Błędy}
\section{Błędy}
{\fontsize{12}{12}\selectfont
    Dane wprowadzane przez użytkownika będą walidowane podczas rejestracji, logowania, a także dodawania i
    edycji ogłoszeń. Jeżeli wystąpi błąd użytkownik zostanie o tym poinformowany. Komunikaty będą wyświetlane m.in. w poniższych sytuacjach:
    
    \begin{enumerate}
        \item zajęta nazwa użytkownika;
        \item hasło nie pasujące do wzorca;
        \item email nie pasujący do wzorca;
        \item cena nie pasujaca do wzorca;
        \item błędne dane logowania;
        \item próba wyświetlenia stron wymagających autoryzacji bez zalogowania;
        \item próba wyświetlenia stron wymagających autentykacji bez posiadania odpowiednich uprawnień.
        
    \end{enumerate}
    
    Błędy nie bedą kończyły działania aplikacji.

}

\end{document}
